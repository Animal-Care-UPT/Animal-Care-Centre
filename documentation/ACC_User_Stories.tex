\documentclass[a4paper,11pt]{article}  

% ------------------ ENCODING & LANGUAGE ------------------
\usepackage[utf8]{inputenc} % UTF-8 encoding
\usepackage[T1]{fontenc}    % Proper font encoding for special characters
\usepackage[portuguese]{babel} % Portuguese hyphenation & translations

% ------------------ PAGE LAYOUT ------------------
\usepackage[left=2cm, right=2cm, top=2cm, bottom=2cm]{geometry} % Custom margins
\usepackage{setspace}   % Control line spacing
\onehalfspacing         % 1.5 line spacing for better readability
\usepackage{parskip}    % Adds space between paragraphs instead of indentation

% ------------------ MATH & SYMBOLS ------------------
\usepackage{amsmath, amssymb, amsthm} % Core math packages
\usepackage{mathtools} % Extends amsmath (e.g., better equation numbering)
\usepackage{bm} % Bold symbols in math mode
\usepackage{siunitx} % SI units, e.g., \SI{9.8}{m/s^2}

% ------------------ GRAPHICS & FIGURES ------------------
\usepackage{graphicx} % Include images
\usepackage{float} % Allow pictures to stay where I put them
\usepackage{xcolor} % Colors for text and diagrams
\usepackage{tikz} % Drawing graphs and diagrams
\usepackage{tikz}
\usetikzlibrary{decorations.pathreplacing}
\usepackage{pgfplots} % Plot functions and data
\pgfplotsset{compat=1.18}

% ------------------ FONT & TEXT IMPROVEMENTS ------------------
\usepackage{lmodern} % Improved font rendering
\usepackage{microtype} % Better spacing and justification

% ------------------ HYPERLINKS ------------------
% \usepackage[colorlinks=true, linkcolor=blue, citecolor=red, urlcolor=blue]{hyperref} 
% Internal links in blue, citations in red, URLs in blue

% ------------------ HEADINGS & CUSTOM STYLES ------------------
\usepackage{titlesec} % Custom section formatting
\titleformat{\section}{\Large\bfseries}{\thesection}{1em}{}
\titleformat{\subsection}{\large\bfseries}{\thesubsection}{1em}{}
\titleformat{\subsubsection}{\normalsize\bfseries}{\thesubsubsection}{1em}{}

% ------------------ TITLE ------------------
\title{Documentação Animal Care Centre}
\author{}
\date{}

\begin{document}

\maketitle
\vspace{-67pt}

\section{Product Backlog}
\subsection{User Story 1}
Como utilizador quero criar uma conta de modo a aceder à plataforma.\\\\
\textbf{Critérios de Aceitação}:
\begin{itemize}
  \item Existe um botão no menu principal para criar conta.
  \item O utilizador deve selecionar o tipo de conta que está a criar.
  \item O cliente tem de introduzir o seu nome, e-mail, palavra-passe, data de nascimento, localização, pergunta secreta e contacto.
  \item A palavra-passe deverá ter entre 8 e 16 caracteres e conter, pelo menos, um caracter especial ou um número.
  \item Caso a palavra-passe não esteja de acordo com os critérios estabelecidos, aparece uma mensagem de erro.
  \item Caso a data de nascimento seja no futuro, aparece uma mensagem de erro.
  \item Caso o e-mail já se encontre registado, aparece uma mensagem de erro.
\end{itemize}

\subsection{User Story 2}
Como administrador quero criar uma conta de modo a aceder à plataforma.\\\\
\textbf{Critérios de Aceitação}:
\begin{itemize}
  \item Existe um botão no menu principal para criar conta.
  \item O administrador deve selecionar o tipo de conta que está a criar.
  \item O cliente tem de introduzir o seu nome, e-mail, palavra-passe, data de nascimento, localização, pergunta secreta e código especial para administrador.
  \item A palavra-passe deverá ter entre 8 e 16 caracteres e conter, pelo menos, um caracter especial ou um número.
  \item Caso a palavra-passe não esteja de acordo com os critérios estabelecidos, aparece uma mensagem de erro.
  \item Caso a data de nascimento seja no futuro, aparece uma mensagem de erro.
  \item Caso o e-mail já se encontre registado, aparece uma mensagem de erro.
  \item Caso o código especial esteja errado, aparece uma mensagem de erro.
\end{itemize}

\subsection{User Story 3}
Como associação quero criar uma conta de modo a aceder à plataforma.\\\\
\textbf{Critérios de Aceitação}:
\begin{itemize}
  \item Existe um botão no menu principal para criar conta.
  \item A associação deve selecionar o tipo de conta que está a criar.
  \item A associação tem de introduzir o seu nome, e-mail, palavra-passe, ano de fundação, localização, pergunta secreta e contacto.
  \item A palavra-passe deverá ter entre 8 e 16 caracteres e conter, pelo menos, um caracter especial ou um número.
  \item Caso a palavra-passe não esteja de acordo com os critérios estabelecidos, aparece uma mensagem de erro.
  \item Caso a data de nascimento seja no futuro, aparece uma mensagem de erro.
  \item Caso o e-mail já se encontre registado, aparece uma mensagem de erro.
  \item A conta fica pendente de verificação de um administrador.
\end{itemize}

\subsection{User Story 4}
Como administrador quero responder aos pedidos de adesão das associações de modo a permitir o seu acesso à plataforma.\\\\
\textbf{Critérios de Aceitação}:
\begin{itemize}
  \item Deve existir uma opção na homepage do administrador que permita aceder aos pedidos de adesão das associações.
  \item Ao aceitar ou rejeitar o pedido, deve aparecer uma mensagem de confirmação.
  \item Caso seja aceite, a conta da associação fica imediatamente disponível.
\end{itemize}

\subsection{User Story 5}
Como administrador quero remover uma associação de modo a manter a confiabilidade da plataforma.\\\\
\textbf{Critérios de Aceitação}:
\begin{itemize}
  \item Deve existir um botão para remover a associação da plataforma na página de pesquisa por associação.
  \item Ao clicar no botão de remoção, o administrador deve introduzir novamente a sua palavra-passe.
  \item Após a remoção, caso a associação faça login na plataforma deve aparecer uma mensagem que informe a mesma de que foi removida da plataforma.
\end{itemize}

\subsection{User Story 6}
Como utilizador quero recuperar a palavra-passe, de modo a manter o acesso à plataforma.\\\\
\textbf{Critérios de Aceitação}:
\begin{itemize}
  \item Existe um botão no menu de login para recuperar a palavra-passe.
  \item Existe um botão na área do utilizador para alterar a palavra-passe.
  \item O utilizador deve introduzir o seu e-mail e a resposta da pergunta secreta.
  \item O utilizador recebe um e-mail com uma hiperligação para alterar a palavra-passe.
  \item A nova palavra-passe deverá ter entre 8 e 16 caracteres e conter, pelo menos, um caracter especial ou um número.
  \item Caso o e-mail não se encontre registado aparece uma mensagem de erro.
  \item Caso a resposta da pergunta secreta esteja errada aparece uma mensagem de erro.
  \item Caso a palavra-passe não esteja de acordo com os critérios estabelecidos aparece uma mensagem de erro.
  \item Após trocar a palavra-passe aparece uma mensagem de sucesso.
\end{itemize}

\subsection{User Story 7}
Como administrador quero recuperar a palavra-passe, de modo a manter o acesso à plataforma.\\\\
\textbf{Critérios de Aceitação}:
\begin{itemize}
  \item Existe um botão no menu de login para recuperar a palavra-passe.
  \item Existe um botão na área do administrador para alterar a palavra-passe.
  \item O administrador deve introduzir o seu e-mail e a resposta da pergunta secreta.
  \item O administrador recebe um e-mail com uma hiperligação para alterar a palavra-passe.
  \item A nova palavra-passe deverá ter entre 8 e 16 caracteres e conter, pelo menos, um caracter especial ou um número.
  \item Caso o e-mail não se encontre registado aparece uma mensagem de erro.
  \item Caso a resposta da pergunta secreta esteja errada aparece uma mensagem de erro.
  \item Caso a palavra-passe não esteja de acordo com os critérios estabelecidos aparece uma mensagem de erro.
  \item Após trocar a palavra-passe aparece uma mensagem de sucesso.
\end{itemize}

\subsection{User Story 8}
Como associação quero recuperar a palavra-passe, de modo a manter o acesso à plataforma.\\\\
\textbf{Critérios de Aceitação}:
\begin{itemize}
  \item Existe um botão no menu de login para recuperar a palavra-passe.
  \item Existe um botão na área da associação para alterar a palavra-passe.
  \item A associação deve introduzir o seu e-mail e a resposta da pergunta secreta.
  \item A associação recebe um e-mail com uma hiperligação para alterar a palavra-passe.
  \item A nova palavra-passe deverá ter entre 8 e 16 caracteres e conter, pelo menos, um caracter especial ou um número.
  \item Caso o e-mail não se encontre registado aparece uma mensagem de erro.
  \item Caso a resposta da pergunta secreta esteja errada aparece uma mensagem de erro.
  \item Caso a palavra-passe não esteja de acordo com os critérios estabelecidos aparece uma mensagem de erro.
  \item Após trocar a palavra-passe aparece uma mensagem de sucesso.
\end{itemize}

\subsection{User Story 9}
Como utilizador quero mudar a minha questão secreta a qualquer momento de modo a garantir que sei a resposta à pergunta.\\\\
\textbf{Critérios de Aceitação}:
\begin{itemize}
  \item Esta opção deverá aparecer na área do utilizador.
  \item De modo a validar esta alteração, o utilizador deve introduzir novamente a sua palavra-pass.
  \item Caso a palavra-passe esteja errada aparece uma mensagem de erro.
  \item Após alterar a questão secreta aparece uma mensagem de sucesso.
\end{itemize}

\subsection{User Story 10}
Como administrador quero mudar a minha questão secreta a qualquer momento de modo a garantir que sei a resposta à pergunta.\\\\
\textbf{Critérios de Aceitação}:
\begin{itemize}
  \item Esta opção deverá aparecer na área do administrador.
  \item De modo a validar esta alteração, o administrador deve introduzir novamente a sua palavra-pass.
  \item Caso a palavra-passe esteja errada aparece uma mensagem de erro.
  \item Após alterar a questão secreta aparece uma mensagem de sucesso.
\end{itemize}

\subsection{User Story 11}
Como associação quero mudar a minha questão secreta a qualquer momento de modo a garantir que sei a resposta à pergunta.\\\\
\textbf{Critérios de Aceitação}:
\begin{itemize}
  \item Esta opção deverá aparecer na área da associação.
  \item De modo a validar esta alteração, a associação deve introduzir novamente a sua palavra-pass.
  \item Caso a palavra-passe esteja errada aparece uma mensagem de erro.
  \item Após alterar a questão secreta aparece uma mensagem de sucesso.
\end{itemize}

\subsection{User Story 12}
Como utilizador quero fazer login na plataforma, de modo a ter acesso às minhas funcionalidades.\\\\
\textbf{Critérios de Aceitação}:
\begin{itemize}
  \item Existe um botão no menu principal para fazer login.
  \item O utilizador deve inserir o seu e-mail e palavra-passe.
  \item O sistema deve mostrar uma mensagem de erro caso a palavra-passe esteja errada.
  \item O sistema deve mostrar uma mensagem de erro caso o e-mail não esteja associado a nenhuma conta.
\end{itemize}

\subsection{User Story 13}
Como administrador quero fazer login na plataforma, de modo a ter acesso às minhas funcionalidades.\\\\
\textbf{Critérios de Aceitação}:
\begin{itemize}
  \item Existe um botão no menu principal para fazer login.
  \item O administrador deve inserir o seu e-mail e palavra-passe.
  \item O sistema deve mostrar uma mensagem de erro caso a palavra-passe esteja errada.
  \item O sistema deve mostrar uma mensagem de erro caso o e-mail não esteja associado a nenhuma conta.
\end{itemize}

\subsection{User Story 14}
Como associação quero fazer login na plataforma, de modo a ter acesso às minhas funcionalidades.\\\\
\textbf{Critérios de Aceitação}:
\begin{itemize}
  \item Existe um botão no menu principal para fazer login.
  \item A associação deve inserir o seu e-mail e palavra-passe.
  \item O sistema deve mostrar uma mensagem de erro caso a palavra-passe esteja errada.
  \item O sistema deve mostrar uma mensagem de erro caso o e-mail não esteja associado a nenhuma conta.
\end{itemize}

\subsection{User Story 15}
Como utilizador quero fazer logout da minha conta de modo a garantir a segurança dos meus dados.\\\\
\textbf{Critérios de Aceitação}:
\begin{itemize}
  \item Existe um botão para fazer logout na área do utilizador.
\end{itemize}

\subsection{User Story 16}
Como administrador quero fazer logout da minha conta de modo a garantir a segurança dos meus dados.\\\\
\textbf{Critérios de Aceitação}:
\begin{itemize}
  \item Existe um botão para fazer logout na área do administrador.
\end{itemize}

\subsection{User Story 17}
Como associação quero fazer logout da minha conta de modo a garantir a segurança dos meus dados.\\\\
\textbf{Critérios de Aceitação}:
\begin{itemize}
  \item Existe um botão para fazer logout na área da associação.
\end{itemize}

\subsection{User Story 18}
Como associação quero responder a pedidos de entrega de animais, de modo a colocá-los para adoção.\\\\
\textbf{Critérios de Aceitação}:
\begin{itemize}
  \item Existe uma opção na homepage associção para visualizar os pedidos de entrega de animais.
  \item A associação deve escolher se o animal fica para adoção ou acolhimento.
  \item Caso o pedido seja aceite, o animal deverá ficar imediatamente disponível na página da associação.
\end{itemize}

\subsection{User Story 19}
Como associação quero registar animais de modo a colocá-los para adoção.\\\\
\textbf{Critérios de Aceitação}:
\begin{itemize}
  \item Existe uma opção na homepage da associção para registar um animal para adoção ou para acolhimento.
  \item O animal deve ser registado com nome, tipo, raça, porte, idade, cor, descrição e imagem.
  \item Existe a opção "não se aplica" para a raça e porte.
  \item Após o registo, o animal fica imediatamente disponível na página da associação.
\end{itemize}

\subsection{User Story 20}
Como associação quero visualizar os animais que tenho, de modo alterar os seus dados.\\\\
\textbf{Critérios de Aceitação}:
\begin{itemize}
  \item Existe uma opção na homepage da associção para visualizar um animal para adoção.
  \item Existe um botão para filtrar os animais que estão para adoção, acolhimento ou todos.
  \item Os animais são dispostos por ordem alfabética de tipo de animal e dentro de cada tipo são organizados do mais antigo para o mais recente.
  \item Existe um botão para organizar os animais por idade (crescente e decrescente dentro de cada tipo) ou ordem alfabética de tipo (crescente ou decrescente).
  \item Cada animal deve mostrar o seu nome, tipo, raça, porte, idade, cor, descrição, histórico, imagem, estado da vacinação, padrinhos, valor recebido em doações e acholhimento/adoção.
  \item A associação pode alterar a idade, descrição, imagem e estado da vacinação.
\end{itemize}

\subsection{User Story 21}
Como associação quero remover um animal que tenho para adoção, de modo a mostrar na minha página apenas animais disponíveis.\\\\
\textbf{Critérios de Aceitação}:
\begin{itemize}
  \item Existe um botão para remover um animal da associação no menu de visualização de animais da associação.
  \item Após remoção, o animal deixa de ser visível para os utilizadores, mas continua visível para a associação.
  \item A associação deve introduzir novamente a sua palavra-passe para confirmar a operação.
\end{itemize}

\subsection{User Story 22}
Como associação quero visualizar as doações recebidas, de modo a gerir os meus fundos.\\\\
\textbf{Critérios de Aceitação}:
\begin{itemize}
  \item Existe uma opção na homepage da associação para ver todas as doações feitas à associação.
  \item Na página de consulta de doações, devem estar listadas da mais recente para a mais antiga.
  \item Cada doação deve mostrar somente a data e o montante.
\end{itemize}

\subsection{User Story 23}
Como utilizador quero solicitar um animal de modo a adotá-lo assim que ficar disponível.\\\\
\textbf{Critérios de Aceitação}:
\begin{itemize}
  \item Existe uma opção na homepage do utilizador que permite solicitar animais para adoção.
  \item O utilizador deve introduzir o tipo de animal, raça, localização e idade de preferência (<3 meses, de 3 meses a 1 ano, 1 a 3 anos, 3 a 5 anos, 5 a 10 anos).
  \item Todos os campos têm a opção "any", caso o utilizador não tenha preferência relativamente a esse campo.
  \item O sistema deve verificar automaticamente se existe algum animal numa associação que satisfaça as condições.
  \item Caso não exista nenhum animal que satifaça as condições, o utilizador fica em lista de espera.
  \item Caso um animal seja registado por uma associação que satisfaça as condições de um utilizador que se encontre em lista de espera, o sistema irá automaticamente fazer o pedido de adoção e o utilizador deve ser notificado.
  \item Após 6 meses em lista de espera a solicitação é removida.
\end{itemize}

\subsection{User Story 24}
Como utilizador quero visualizar os meus pedidos de adoção pendentes, de modo a verificar o seu estado.\\\\
\textbf{Critérios de Aceitação}:
\begin{itemize}
  \item Existe uma secção com os pedidos de adoção e solicitações pendentes na área do utilizador.
\end{itemize}

\subsection{User Story 25}
Como utilizador quero visualizar os meus pedidos de adoção pendentes, de modo a verificar o seu estado.\\\\
\textbf{Critérios de Aceitação}:
\begin{itemize}
  \item Existe um botão na área do utilizador que permite cancelar os pedidos de adoção e solicitações de animais pendentes.
  \item Caso o utilizador tente cancelar, aparece um popup para confirmar o cancelamento.
\end{itemize}

\subsection{User Story 26}
Como administrador quero visualizar os apadrinhamentos de modo a supervisionar as operações da plataforma.\\\\
\textbf{Critérios de Aceitação}:
\begin{itemize}
  \item Existe um botão na homepage do administrador para visualizar todos os apadrinhamentos de todos os animais que se encontrem na plataforma.
  \item Cada apadrinhamento deve mostrar o nome e tipo do animal, nome da associação à qual o animal pertence e o nome do padrinho.
\end{itemize}

\subsection{User Story 27}
Como utilizador quero procurar animais de modo a poder adotar um animal.\\\\
\textbf{Critérios de Aceitação}:
\begin{itemize}
  \item No cabeçalho do portal existe um botão para pesquisar por animal.
  \item Ao clicar no botão de pesquisa por animal, aparece uma janela de pesquisa.
  \item A janela de pesquisa inclui uma caixa de pesquisa, filtros e um botão para pesquisar.
  \item Os filtros são tipo, raça, porte, região, idade, estado (para adoção, para acolhimento ou ambos).
  \item Devem aparecer apenas animais que se encontrem disponíveis.
  \item A listagem de animais deve aparecer, por defeito, por ordem afabética de tipo e, dentro do tipo, dos adicionados há mais tempo para os adicionados mais recentemente.
  \item Deve ser possível entrar no perfil do animal para ver informações adicionais ou pedir para adotar/acolher.
\end{itemize}

\subsection{User Story 28}
Como administrador quero procurar animais de modo a ver as suas informações.\\\\
\textbf{Critérios de Aceitação}:
\begin{itemize}
  \item No cabeçalho do portal existe um botão para pesquisar por animal.
  \item Ao clicar no botão de pesquisa por animal, aparece uma janela de pesquisa.
  \item A janela de pesquisa inclui uma caixa de pesquisa, filtros e um botão para pesquisar.
  \item Os filtros são tipo, raça, porte, região, idade, estado (para adoção, para acolhimento ou ambos).
  \item Devem aparecer apenas animais que se encontrem disponíveis.
  \item A listagem de animais deve aparecer, por defeito, por ordem afabética de tipo e, dentro do tipo, dos adicionados há mais tempo para os adicionados mais recentemente.
  \item Deve ser possível entrar no perfil do animal para ver informações adicionais.
\end{itemize}

\subsection{User Story 29}
Como utilizador quero procurar associações de modo a visualizar o seu conteúdo.\\\\
\textbf{Critérios de Aceitação}:
\begin{itemize}
  \item No cabeçalho do portal existe um botão para pesquisar por associação.
  \item Ao clicar no botão de pesquisa por associação, aparece uma janela de pesquisa.
  \item A janela de pesquisa inclui uma caixa de pesquisa, filtro de região e um botão para pesquisar.
  \item A listagem de associações deve aparecer, por defeito, por ordem afabética de região e, dentro da região, por ordem alfabética de nome.
  \item Deve ser possível entrar no perfil da associação para ver informações adicionais ou doar fundos.
\end{itemize}

\subsection{User Story 30}
Como administrador quero procurar associações de modo a visualizar o seu conteúdo.\\\\
\textbf{Critérios de Aceitação}:
\begin{itemize}
  \item No cabeçalho do portal existe um botão para pesquisar por associação.
  \item Ao clicar no botão de pesquisa por associação, aparece uma janela de pesquisa.
  \item A janela de pesquisa inclui uma caixa de pesquisa, filtro de região e um botão para pesquisar.
  \item A listagem de associações deve aparecer, por defeito, por ordem afabética de região e, dentro da região, por ordem alfabética de nome.
  \item Deve ser possível entrar no perfil da associação para ver informações adicionais.
\end{itemize}

\subsection{User Story 31}
Como utilizador quero fazer um pedido para adoção de modo a adotar um animal.\\\\
\textbf{Critérios de Aceitação}:
\begin{itemize}
  \item O animal deve-se encontrar disponivel para adoção.
  \item Deve existir uma opção no perfil do animal "Adotar".
  \item Enquanto o pedido não for aceite ou rejeitado, este deve aparecer na área do utilizador.
  \item Deve ser possivel cancelar um pedido de adoção através de um botão na área do utilizador.
  \item O utilizador deve ser notificado da resposta ao pedido de adoção.
\end{itemize}

\subsection{User Story 32}
Como utlizador quero ter acesso ao meu historico de adoções de modo a visualizar todas as adoções que já fiz.\\\\
\textbf{Critérios de Aceitação}:
\begin{itemize}
  \item Deve existir um botão na homepage do utilizador a dizer historico adoções.
  \item O historico deve estar ordenado pela data de adoção.
\end{itemize}

\subsection{User Story 33}
Como utilizador quero poder apadrinhar um animal de modo a fazer uma doação periodica.\\\\
\textbf{Critérios de Aceitação}:
\begin{itemize}
  \item O animal deve estar disponivel para apadrinhamento.
  \item Um animal pode ter no maximo 3 padrinhos.
  \item A doação deve ter um valor minimo mensal de 5€ e um máximo de 1000€.
  \item Deve existir uma opção no perfil do animal a dizer apadrinhar.
  \item Se o limite for atingido, o botão “Apadrinhar” deve ficar indisponível.
\end{itemize}

\subsection{User Story 34}
Como utilizador quero ter acesso ao meu historico de doações de modo a visualizar o quanto já doei.\\\\
\textbf{Critérios de Aceitação}:
\begin{itemize}
  \item Existe um botão na homepage do utilizador que leva à página do histórico de doações.
  \item A página de doações deve ter duas secções: apadrinhamentos e doações.
  \item Na secção de apadrinhamentos devem ser visíveis os animais que o utilizador apadrinha atualmente, assim como, todos os apadrinhamentos anteriores, com datas de início ao fim.
  \item Na secção das doações devem aparecer todas as doações feitas pelo utilizador, incluindo as mensalidades de apadrinhamentos.
  \item Na doação deve aparecer a data, montante e o tipo de doação.
  \item Deve aparecer no maximo 30 doações por pagina.
  \item Cada doação deve identificar o nome da associação ou o nome do animal para quem foi doado.
  \item Se o limite for atingido, o botão “Apadrinhar” deve ficar indisponível.
  \item Deve aparecer uma mensagem antes da lista de doações a agradecer o contributo da pessoa e a informar do total que ja foi doado.
\end{itemize}

\subsection{User Story 35}
Como utilizador quero poder cancelar o apadrinhamento a qualquer momento de modo a parar de pagar o montante mensal.\\\\
\textbf{Critérios de Aceitação}:
\begin{itemize}
  \item Deve existir um botão na pagina de doações que permite cancelar um apadrinhamento atual.
  \item Ao clicar para cancelar apadrinhamento o mesmo é automaticamente cancelado,
\end{itemize}

\end{document}


