\documentclass[a4paper,11pt]{article}  

% ------------------ ENCODING & LANGUAGE ------------------
\usepackage[utf8]{inputenc} % UTF-8 encoding
\usepackage[T1]{fontenc}    % Proper font encoding for special characters
\usepackage[portuguese]{babel} % Portuguese hyphenation & translations

% ------------------ PAGE LAYOUT ------------------
\usepackage[left=2cm, right=2cm, top=2cm, bottom=2cm]{geometry} % Custom margins
\usepackage{setspace}   % Control line spacing
\onehalfspacing         % 1.5 line spacing for better readability
\usepackage{parskip}    % Adds space between paragraphs instead of indentation

% ------------------ MATH & SYMBOLS ------------------
\usepackage{amsmath, amssymb, amsthm} % Core math packages
\usepackage{mathtools} % Extends amsmath (e.g., better equation numbering)
\usepackage{bm} % Bold symbols in math mode
\usepackage{siunitx} % SI units, e.g., \SI{9.8}{m/s^2}

% ------------------ GRAPHICS & FIGURES ------------------
\usepackage{graphicx} % Include images
\usepackage{float} % Allow pictures to stay where I put them
\usepackage{xcolor} % Colors for text and diagrams
\usepackage{tikz} % Drawing graphs and diagrams
\usepackage{tikz}
\usetikzlibrary{decorations.pathreplacing}
\usepackage{pgfplots} % Plot functions and data
\pgfplotsset{compat=1.18}

% ------------------ FONT & TEXT IMPROVEMENTS ------------------
\usepackage{lmodern} % Improved font rendering
\usepackage{microtype} % Better spacing and justification

% ------------------ HYPERLINKS ------------------
% \usepackage[colorlinks=true, linkcolor=blue, citecolor=red, urlcolor=blue]{hyperref} 
% Internal links in blue, citations in red, URLs in blue

% ------------------ HEADINGS & CUSTOM STYLES ------------------
\usepackage{titlesec} % Custom section formatting
\titleformat{\section}{\Large\bfseries}{\thesection}{1em}{}
\titleformat{\subsection}{\large\bfseries}{\thesubsection}{1em}{}
\titleformat{\subsubsection}{\normalsize\bfseries}{\thesubsubsection}{1em}{}

% ------------------ TITLE ------------------
\title{Documentação Animal Care Centre}
\author{}
\date{}

\begin{document}

\maketitle
\vspace{-67pt}

\section{Product Backlog}
\subsection{User Story 1}
Como user quero criar uma conta de modo a aceder à plataforma.\\\\
\textbf{Critérios de Aceitação}:
\begin{itemize}
  \item Existe um botão no menu principal para criar conta.
  \item O user deve selecionar o tipo de conta que está a criar.
  \item O cliente tem de introduzir o seu nome, e-mail, palavra-passe, data de nascimento, localização e pergunta secreta.
  \item A palavra-passe deverá ter entre 8 e 16 caracteres e conter, pelo menos, um caracter especial ou um número.
  \item Caso a palavra-passe não esteja de acordo com os critérios estabelecidos, aparece uma mensagem de erro.
  \item Caso a data de nascimento seja no futuro, aparece uma mensagem de erro.
  \item Caso o e-mail já se encontre registado, aparece uma mensagem de erro.
\end{itemize}

\subsection{User Story 2}
Como admin quero criar uma conta de modo a aceder à plataforma.\\\\
\textbf{Critérios de Aceitação}:
\begin{itemize}
  \item Existe um botão no menu principal para criar conta.
  \item O admin deve selecionar o tipo de conta que está a criar.
  \item O cliente tem de introduzir o seu nome, e-mail, palavra-passe, data de nascimento, localização, pergunta secreta e código especial para admin.
  \item A palavra-passe deverá ter entre 8 e 16 caracteres e conter, pelo menos, um caracter especial ou um número.
  \item Caso a palavra-passe não esteja de acordo com os critérios estabelecidos, aparece uma mensagem de erro.
  \item Caso a data de nascimento seja no futuro, aparece uma mensagem de erro.
  \item Caso o e-mail já se encontre registado, aparece uma mensagem de erro.
  \item Caso o código especial esteja errado, aparece uma mensagem de erro.
\end{itemize}

\subsection{User Story 3}
Como associação quero criar uma conta de modo a aceder à plataforma.\\\\
\textbf{Critérios de Aceitação}:
\begin{itemize}
  \item Existe um botão no menu principal para criar conta.
  \item A associação deve selecionar o tipo de conta que está a criar.
  \item A associação tem de introduzir o seu nome, e-mail, palavra-passe, ano de fundação, localização e pergunta secreta.
  \item A palavra-passe deverá ter entre 8 e 16 caracteres e conter, pelo menos, um caracter especial ou um número.
  \item Caso a palavra-passe não esteja de acordo com os critérios estabelecidos, aparece uma mensagem de erro.
  \item Caso a data de nascimento seja no futuro, aparece uma mensagem de erro.
  \item Caso o e-mail já se encontre registado, aparece uma mensagem de erro.
  \item A conta fica pendente de verificação de um admin.
\end{itemize}

\subsection{User Story 4}
Como admin quero responder aos pedidos de adesão das associações de modo a permitir o seu acesso à plataforma.\\\\
\textbf{Critérios de Aceitação}:
\begin{itemize}
  \item Deve existir uma opção na homepage do administrador que permita aceder aos pedidos de adesão das associações.
  \item Ao aceitar ou rejeitar o pedido, deve aparecer uma mensagem de confirmação.
  \item Caso seja aceite, a conta da associação fica imediatamente disponível.
\end{itemize}

\subsection{User Story 5}
Como admin quero remover uma associação de modo a manter a confiabilidade da plataforma.\\\\
\textbf{Critérios de Aceitação}:
\begin{itemize}
  \item Deve existir uma opção na homepage do administrador que mostre todas as associações registadas.
  \item Deve existir um botão para remover a associação da plataforma.
  \item Ao clicar no botão de remoção, o administrador deve introduzir novamente a sua palavra-passe.
  \item Após a remoção, caso a associação faça login na plataforma deve aparecer uma mensagem que informe a mesma de que foi removida da plataforma.
\end{itemize}

\subsection{User Story 6}
Como user quero recuperar a palavra-passe, de modo a manter o acesso à plataforma.\\\\
\textbf{Critérios de Aceitação}:
\begin{itemize}
  \item Existe um botão no menu de login para recuperar a palavra-passe.
  \item O user deve introduzir o seu e-mail e a resposta da pergunta secreta.
  \item O user recebe um e-mail com uma hiperligação para alterar a palavra-passe.
  \item A nova palavra-passe deverá ter entre 8 e 16 caracteres e conter, pelo menos, um caracter especial ou um número.
  \item Caso o e-mail não se encontre registado aparece uma mensagem de erro.
  \item Caso a resposta da pergunta secreta esteja errada aparece uma mensagem de erro.
  \item Caso a palavra-passe não esteja de acordo com os critérios estabelecidos aparece uma mensagem de erro.
  \item Após trocar a palavra-passe aparece uma mensagem de sucesso.
\end{itemize}

\subsection{User Story 7}
Como admin quero recuperar a palavra-passe, de modo a manter o acesso à plataforma.\\\\
\textbf{Critérios de Aceitação}:
\begin{itemize}
  \item Existe um botão no menu de login para recuperar a palavra-passe.
  \item O admin deve introduzir o seu e-mail e a resposta da pergunta secreta.
  \item O admin recebe um e-mail com uma hiperligação para alterar a palavra-passe.
  \item A nova palavra-passe deverá ter entre 8 e 16 caracteres e conter, pelo menos, um caracter especial ou um número.
  \item Caso o e-mail não se encontre registado aparece uma mensagem de erro.
  \item Caso a resposta da pergunta secreta esteja errada aparece uma mensagem de erro.
  \item Caso a palavra-passe não esteja de acordo com os critérios estabelecidos aparece uma mensagem de erro.
  \item Após trocar a palavra-passe aparece uma mensagem de sucesso.
\end{itemize}

\subsection{User Story 8}
Como associação quero recuperar a palavra-passe, de modo a manter o acesso à plataforma.\\\\
\textbf{Critérios de Aceitação}:
\begin{itemize}
  \item Existe um botão no menu de login para recuperar a palavra-passe.
  \item A associação deve introduzir o seu e-mail e a resposta da pergunta secreta.
  \item A associação recebe um e-mail com uma hiperligação para alterar a palavra-passe.
  \item A nova palavra-passe deverá ter entre 8 e 16 caracteres e conter, pelo menos, um caracter especial ou um número.
  \item Caso o e-mail não se encontre registado aparece uma mensagem de erro.
  \item Caso a resposta da pergunta secreta esteja errada aparece uma mensagem de erro.
  \item Caso a palavra-passe não esteja de acordo com os critérios estabelecidos aparece uma mensagem de erro.
  \item Após trocar a palavra-passe aparece uma mensagem de sucesso.
\end{itemize}

\subsection{User Story 9}
Como user quero mudar a minha questão secreta a qualquer momento de modo a garantir que sei a resposta à pergunta.\\\\
\textbf{Critérios de Aceitação}:
\begin{itemize}
  \item Esta opção deverá aparecer na área do user.
  \item De modo a validar esta alteração, o user deve introduzir novamente a sua palavra-pass.
  \item Caso a palavra-passe esteja errada aparece uma mensagem de erro.
  \item Após alterar a questão secreta aparece uma mensagem de sucesso.
\end{itemize}

\subsection{User Story 10}
Como admin quero mudar a minha questão secreta a qualquer momento de modo a garantir que sei a resposta à pergunta.\\\\
\textbf{Critérios de Aceitação}:
\begin{itemize}
  \item Esta opção deverá aparecer na área do admin.
  \item De modo a validar esta alteração, o admin deve introduzir novamente a sua palavra-pass.
  \item Caso a palavra-passe esteja errada aparece uma mensagem de erro.
  \item Após alterar a questão secreta aparece uma mensagem de sucesso.
\end{itemize}

\subsection{User Story 11}
Como associação quero mudar a minha questão secreta a qualquer momento de modo a garantir que sei a resposta à pergunta.\\\\
\textbf{Critérios de Aceitação}:
\begin{itemize}
  \item Esta opção deverá aparecer na área da associação.
  \item De modo a validar esta alteração, a associação deve introduzir novamente a sua palavra-pass.
  \item Caso a palavra-passe esteja errada aparece uma mensagem de erro.
  \item Após alterar a questão secreta aparece uma mensagem de sucesso.
\end{itemize}

\subsection{User Story 12}
Como user quero fazer login na plataforma, de modo a ter acesso às minhas funcionalidades.\\\\
\textbf{Critérios de Aceitação}:
\begin{itemize}
  \item Existe um botão no menu principal para fazer login.
  \item O user deve inserir o seu e-mail e palavra-passe.
  \item O sistema deve mostrar uma mensagem de erro caso a palavra-passe esteja errada.
  \item O sistema deve mostrar uma mensagem de erro caso o e-mail não esteja associado a nenhuma conta.
\end{itemize}

\subsection{User Story 13}
Como admin quero fazer login na plataforma, de modo a ter acesso às minhas funcionalidades.\\\\
\textbf{Critérios de Aceitação}:
\begin{itemize}
  \item Existe um botão no menu principal para fazer login.
  \item O admin deve inserir o seu e-mail e palavra-passe.
  \item O sistema deve mostrar uma mensagem de erro caso a palavra-passe esteja errada.
  \item O sistema deve mostrar uma mensagem de erro caso o e-mail não esteja associado a nenhuma conta.
\end{itemize}

\subsection{User Story 14}
Como associação quero fazer login na plataforma, de modo a ter acesso às minhas funcionalidades.\\\\
\textbf{Critérios de Aceitação}:
\begin{itemize}
  \item Existe um botão no menu principal para fazer login.
  \item A associação deve inserir o seu e-mail e palavra-passe.
  \item O sistema deve mostrar uma mensagem de erro caso a palavra-passe esteja errada.
  \item O sistema deve mostrar uma mensagem de erro caso o e-mail não esteja associado a nenhuma conta.
\end{itemize}

\subsection{User Story 15}
Como user quero fazer logout da minha conta de modo a garantir a segurança dos meus dados.\\\\
\textbf{Critérios de Aceitação}:
\begin{itemize}
  \item Existe um botão para fazer logout na área do user.
\end{itemize}

\subsection{User Story 16}
Como admin quero fazer logout da minha conta de modo a garantir a segurança dos meus dados.\\\\
\textbf{Critérios de Aceitação}:
\begin{itemize}
  \item Existe um botão para fazer logout na área do admin.
\end{itemize}

\subsection{User Story 17}
Como associação quero fazer logout da minha conta de modo a garantir a segurança dos meus dados.\\\\
\textbf{Critérios de Aceitação}:
\begin{itemize}
  \item Existe um botão para fazer logout na área da associação.
\end{itemize}

\subsection{User Story 18}
Como associação quero responder a pedidos de entrega de animais, de modo a colocá-los para adoção.\\\\
\textbf{Critérios de Aceitação}:
\begin{itemize}
  \item Existe uma opção na homepage associção para visualizar os pedidos de entrega de animais.
  \item A associação deve escolher se o animal fica para adoção ou acolhimento.
  \item Caso o pedido seja aceite, o animal deverá ficar imediatamente disponível na página da associação.
\end{itemize}

\subsection{User Story 19}
Como associação quero registar animais de modo a colocá-los para adoção.\\\\
\textbf{Critérios de Aceitação}:
\begin{itemize}
  \item Existe uma opção na homepage da associção para registar um animal para adoção.
  \item O animal deve ser registado com nome, tipo, raça (opcional), porte (opcional), idade, cor, descrição e imagem.
  \item Após o registo, o animal fica imediatamente disponível na página da associação.
\end{itemize}

\subsection{User Story 20}
Como associação quero visualizar os animais que tenho para adoção, de modo alterar os seus dados.\\\\
\textbf{Critérios de Aceitação}:
\begin{itemize}
  \item Existe uma opção na homepage da associção para visualizar um animal para adoção.
  \item Os animais são dispostos por ordem alfabética de tipo de animal e dentro de cada tipo são organizados por ordem alfabética de raça.
  \item Existe um botão para organizar os animais por idade (crescente e decrescente dentro de cada tipo) ou ordem alfabética de tipo (crescente ou decrescente).
  \item Cada animal deve mostrar o seu nome, tipo, raça, porte, idade, cor, descrição, histórico, imagem, estado da vacinação, padrinhos, valor recebido em doações e acholhimento/adoção.
  \item A associação pode alterar a idade, descrição, imagem e estado da vacinação.
\end{itemize}

\subsection{User Story 21}
Como associação quero remover um animal que tenho para adoção, de modo a mostrar na minha página apenas animais disponíveis.\\\\
\textbf{Critérios de Aceitação}:
\begin{itemize}
  \item Existe um botão para remover um animal da associação no menu de visualização de animais da associação.
  \item A associação deve introduzir novamente a sua palavra-passe para confirmar a operação.
\end{itemize}

\subsection{User Story 22}
Como associação quero visualizar as doações recebidas, de modo a gerir os meus fundos.\\\\
\textbf{Critérios de Aceitação}:
\begin{itemize}
  \item Existe uma opção na homepage da associação para ver todas as doações feitas à associação.
  \item Cada doação deve mostrar o doador, a data e o montante.
\end{itemize}

\subsection{User Story 23}
Como user quero solicitar um animal de modo a adotá-lo assim que ficar disponível.\\\\
\textbf{Critérios de Aceitação}:
\begin{itemize}
  \item Existe uma opção na homepage do user que permite solicitar animais para adoção.
  \item O user deve introduzir o tipo de animal, raça, localização e idade de preferência (<3 meses, de 3 meses a 1 ano, 1 a 3 anos, 3 a 5 anos, 5 a 10 anos).
  \item Todos os campos têm a opção "any", caso o user não tenha preferência relativamente a esse campo.
  \item O sistema deve verificar automaticamente se existe algum animal numa associação que satisfaça as condições.
  \item Caso não exista nenhum animal que satifaça as condições, o user fica em lista de espera.
  \item Caso um animal seja registado por uma associação que satisfaça as condições de um user que se encontre em lista de espera, o sistema irá automaticamente fazer o pedido de adoção e o user deve ser notificado.
  \item Após 6 meses em lista de espera a solicitação é removida.
\end{itemize}

\subsection{User Story 24}
Como user quero visualizar os meus pedidos de adoção pendentes, de modo a verificar o seu estado.
\textbf{Critérios de Aceitação}:
\begin{itemize}
  \item Existe uma secção com os pedidos de adoção e solicitações pendentes na área do user.
\end{itemize}

\subsection{User Story 25}
Como user quero visualizar os meus pedidos de adoção pendentes, de modo a verificar o seu estado.
\textbf{Critérios de Aceitação}:
\begin{itemize}
  \item Existe um botão na área do user que permite cancelar os pedidos de adoção e solicitações de animais pendentes.
  \item Caso o user tente cancelar, aparece um popup para confirmar o cancelamento.
\end{itemize}

\end{document}


